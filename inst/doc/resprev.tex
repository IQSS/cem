
Dear Jan:

Thanks for the opportunity to revise our paper, and for the reviews.
The new version of our submission is in the same place:
http://gking.harvard.edu/cem.  (We will move it to JSS format if its
accepted of course.)  We summarize our changes below:

Neither reviewer asks for changes in the software and so, aside from
some general improvements and bug fixes, we haven't changed the
software.

R1's comments:

1. As per R1's excellent suggestion, we have expanded the introductory
section into a complete summary of what we're doing.  The subsections
cover an overview, the properties, the goal, the algorithm, and
measuring balance (including the L1 statistic).  This addition means
that this paper is now an entirely self-contained in terms of a
summary of what our package does.  One no longer needs to read our
unpublished paper for anything other than proofs of the properties of
CEM.  Perhaps this will help too with some of Ben Hansen's remarks.

2. On R1's other point, given the time it takes papers to get
accepted, we would prefer not to wait until our other paper is
published, but we will certainly keep updating our JSS submission as
we learn things from other referee processes or elsewhere.  (We
appreciate his or her note about the "bludgeoning those who propose
new matching estimators," a topic to which we now turn!  :-). )

On Ben Hansen's points:

As suggested, we have toned down our language.  We still have one
subsection in the introduction that lists the properties of CEM that
are proven in our other paper.  Reasonable people can differ about
whether the properties of CEM are of interest in actual data analysis,
but the proofs in our other paper are correct (and no one has claimed
otherwise), and so the properties listed in this paper are accurate.
We have tried to portray them in a more unobjectionable factual way.

We have included Ben's point that although CEM bounds the maximum
imbalance (as well as the causal effect estimation error and degree of
model dependence), the actual imbalance achieved may be less than the
maximum.  This is a good point and it is the reason we also introduce
a new more comprehensive measure of multivariate imbalance.  We now
address these two points together in the introduction.

We explain more clearly in the introduction of this paper (and in a
separate section of our companion paper by the way) that any method of
matching can be applied within CEM strata and it will inherit all of
CEM's properties.  This includes propensity scores that Ben is
concerned about, as well as all other proposed methods we know of.

We followed Ben's advice on his suggestion marked ``p.4''.

``p.6'': We have removed the word ``full'' as suggested.  In addition,
as we now describe more clearly in the revised version of our paper,
our L_1 measure depends on a bin size that is fixed prior to applying
CEM or any other method of matching.  We set the bin size via
automatic methods used for drawing histograms, and set it so that the
bins are smaller than those typically chosen by users for levels of
coarsening in CEM.

``p.8'': We added the suggested information.

``p.10'': The claim in the quote Ben includes from our paper is
accurate.  Ben may think it is ``bombast'' but it is in fact true; the
proof is a real proof.  We also think Ben misinterprets the point: it
is not merely the bounding that is relevant here, but the fact that
the user -- by setting the coarsening before matching --- determines
this maximum imbalance and completely controls it.  Although other
methods of matching are also bounded in the trivial way Ben describes,
almost all do not let the user set this maximum acceptable level of
imbalance ex ante.  For most of these other methods, the only way
balance is improved is as a side effect of the method, and for the
vast majority there is no guarantee that balance will be improved at
all (in practice, most often, balance gets worse on some variables
while it improves on others).  It might be nicer to set the actual
balance ex ante, but that method does not exist; we think that in
practice being able to totally control the maximum imbalance is a
tremendous convenience, especially since changing the maximum
imbalance on one variable has no effect on the others.  In all the
empirical examples we analyze, including the ones in the present
paper, we find it far, far easier to achieve balance with CEM than
with methods that only enable you to check balance after the fact.  We
are convinced that applied users have ``exhaust [their] patience'' far
faster with methods that do not have this property, but whether or not
one finds this property useful may not be so material for the present
paper, since the property is clearly true, our description of it is
accurate, and the software does implement it as indicated.  We have
however tried to tone down our excitement for it, and just keep to the
facts.



%%% Local Variables: 
%%% mode: latex
%%% TeX-master: t
%%% End: 
