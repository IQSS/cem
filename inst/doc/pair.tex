\subsection{{\tt pair}: Produces a paired sample out of a CEM match solution}\label{ss:pair}
\keyword{multivariate}{pair}
\keyword{datagen}{pair}
\begin{Description}\relax
Produces a paired sample out of a CEM match solution
\end{Description}
\begin{Usage}
\begin{verbatim}
pair(obj, data, method=NULL, mpower=2, verbose=0)
\end{verbatim}
\end{Usage}
\begin{Arguments}
\begin{ldescription}
\item[\code{obj}] an object as output from \code{cem}
\item[\code{data}] the original data.frame used by \code{cem}
\item[\code{method}] distance method to use in \code{k2k} matching. See Details.
\item[\code{mpower}] power of the Minkowski distance. See Details.
\item[\code{verbose}] controls level of verbosity. Default=0.
\end{ldescription}
\end{Arguments}
\begin{Details}\relax
This function returns a vector of paired matched units index.

The user can choose a \code{method} (between `\code{euclidean}',
`\code{maximum}', `\code{manhattan}', `\code{canberra}', `\code{binary}'
and `\code{minkowski}') for nearest neighbor matching inside each
\code{cem} strata. By default \code{method} is set to `\code{NULL}',
which means random matching inside \code{cem} strata. For the Minkowski
distance the power can be specified via the argument \code{mpower}'.
For more information on \code{method != NULL}, refer to
\code{\LinkA{dist}{dist}} help page.
\end{Details}
\begin{Value}
\begin{ldescription}
\item[\code{obj}] a list with the fields \code{paired}, \code{full.paired},
\code{reservoir} and \code{reservoir2}. The latter contain the indexes
of the unmatched units.
\end{ldescription}
\end{Value}
\begin{Author}\relax
Stefano Iacus, Gary King, and Giuseppe Porro
\end{Author}
\begin{References}\relax
Stefano Iacus, Gary King, Giuseppe Porro, ``Matching for
Casual Inference Without Balance Checking,''
http://gking.harvard.edu/files/abs/cem-abs.shtml
\end{References}
\begin{Examples}
\begin{ExampleCode}
data(LL)

# cem match: automatic bin choice
mat <- cem(data=LL, drop="re78")

# we want a set of paired units
psample <- pair(mat, data=LL)
table(psample$paired)
psample$paired[1:100]

table(psample$full.paired)
psample$full.paired[1:10]


# cem match: automatic bin choice, we drop one row from the data set
mat1 <- cem(data=LL[-1,], drop="re78")

# we want a set of paired units but we have an odd number of units in the data
psample <- pair(mat1, data=LL[-1,])
table(psample$full.paired)
\end{ExampleCode}
\end{Examples}

